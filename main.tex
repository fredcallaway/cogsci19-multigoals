% 
% Annual Cognitive Science Conference
% Sample LaTeX Paper -- Proceedings Format
% 

% Original : Ashwin Ram (ashwin@cc.gatech.edu)       04/01/1994
% Modified : Johanna Moore (jmoore@cs.pitt.edu)      03/17/1995
% Modified : David Noelle (noelle@ucsd.edu)          03/15/1996
% Modified : Pat Langley (langley@cs.stanford.edu)   01/26/1997
% Latex2e corrections by Ramin Charles Nakisa        01/28/1997 
% Modified : Tina Eliassi-Rad (eliassi@cs.wisc.edu)  01/31/1998
% Modified : Trisha Yannuzzi (trisha@ircs.upenn.edu) 12/28/1999 (in process)
% Modified : Mary Ellen Foster (M.E.Foster@ed.ac.uk) 12/11/2000
% Modified : Ken Forbus                              01/23/2004
% Modified : Eli M. Silk (esilk@pitt.edu)            05/24/2005
% Modified : Niels Taatgen (taatgen@cmu.edu)         10/24/2006
% Modified : David Noelle (dnoelle@ucmerced.edu)     11/19/2014

%% Change "letterpaper" in the following line to "a4paper" if you must.

\documentclass[10pt,letterpaper]{article}

\usepackage{cogsci}
\usepackage{pslatex}
\usepackage[natbibapa]{apacite}
\usepackage{amsmath}
\usepackage{amssymb}


\title{Limitations on goal representation in hierarchical planning}
 
\author{Anon Y. Mous}


\begin{document}

\maketitle


\begin{abstract}
When faced with a large and complex problem, people naturally break it up into several smaller and simpler problems. This hierarchical decomposition of an ultimate goal into sub-goals facilitates planning by reducing the number of factors that must be considered at one time. However, it can also lead to suboptimal decision making, leading one to miss opportunities to make progress towards multiple subgoals with a single action. This potential for suboptimality can be ameliorated by considering multiple subgoals at once, but at the cost of increasing the number of factors that must be considered and thus increasing the computational burden of planning. Here, we present a model of planning with compositional goal representations and show that it explains the errors people make in a Towers of London task better than a limited-depth search model. Our results suggest that people are capable of representing and pursuing multiple subgoals at once, but that the number of subgoals is generally quite limited. Furthermore, we find that the degree of representational limitation may be an important latent dimension for explaining individual differences in planning ability. Finally, we (hopefully!) find that the inferred representational limitations are sensitive to the cost of suboptimal planning, indicating that people may rationally choose a goal representation that trades off between representational costs and decision quality.

facilitates planning by allowing the 

\textbf{Keywords:} 
planning; hierarchy; goals
\end{abstract}


\section{Introduction}

Motivating example.

Background on problem solving. Introduce Towers of Hanoi/London. Introduce idea of limited depth planning.

Background on hierarchical representation and subgoals.

Compositional goals/options.

Overview of present work.


\section{Model}
We model participant decisions as emerging from a two step hierarchical planning procedure. 

- Limited depth model

- Limited goal representation model



\section{Experiment}

\subsection{Methods}
\subsubsection{Stimuli and procedure}
\subsubsection{Participants}

\subsection{Results}
Long section.


\section{Discussion}
Why does this matter? What are next steps?


\bibliographystyle{apacite}
\setlength{\bibleftmargin}{.125in}
\setlength{\bibindent}{-\bibleftmargin}

\bibliography{references}


\end{document}
